% -*- latex -*-

%%%%%%%%%%%%%%%%%%%%%%%%%%%%%%%%%%%%%%%%%%%%%%%%%%%%%%%%%%%%%%%%%%%%%%%%%%%%%
% This beginning part of the preamble is psecific to the IEEEtran document
% class.

\documentclass[conference]{IEEEtran}

\author{
  \IEEEauthorblockN{Kenneth~Moreland}
  \IEEEauthorblockA{Sandia National Laboratories\\
    Albuquerque, NM 87185-1326\\
    Email: kmorel@sandia.gov}
  \and
  \IEEEauthorblockN{Others}
  \IEEEauthorblockA{Cool Group\\
    Somewhere, WQ 12341\\
    Email: person@myspace.com}
}

% for over three affiliations, or if they all won't fit within the width
% of the page, use this alternative format:
% 
%\author{\IEEEauthorblockN{Michael Shell\IEEEauthorrefmark{1},
%Homer Simpson\IEEEauthorrefmark{2},
%James Kirk\IEEEauthorrefmark{3}, 
%Montgomery Scott\IEEEauthorrefmark{3} and
%Eldon Tyrell\IEEEauthorrefmark{4}}
%\IEEEauthorblockA{\IEEEauthorrefmark{1}School of Electrical and Computer Engineering\\
%Georgia Institute of Technology,
%Atlanta, Georgia 30332--0250\\ Email: see http://www.michaelshell.org/contact.html}
%\IEEEauthorblockA{\IEEEauthorrefmark{2}Twentieth Century Fox, Springfield, USA\\
%Email: homer@thesimpsons.com}
%\IEEEauthorblockA{\IEEEauthorrefmark{3}Starfleet Academy, San Francisco, California 96678-2391\\
%Telephone: (800) 555--1212, Fax: (888) 555--1212}
%\IEEEauthorblockA{\IEEEauthorrefmark{4}Tyrell Inc., 123 Replicant Street, Los Angeles, California 90210--4321}}

% End of IEEEtran-specific portion of the preamble.
%%%%%%%%%%%%%%%%%%%%%%%%%%%%%%%%%%%%%%%%%%%%%%%%%%%%%%%%%%%%%%%%%%%%%%%%%%%%%


\usepackage{amsfonts}
\usepackage{amssymb}
\usepackage{amsmath}
\usepackage{graphicx}
\usepackage{varioref}
\usepackage{fancyvrb}
\usepackage{ifthen}
\usepackage{cite}
\usepackage{subfig}
\usepackage{xspace}
\usepackage[pdfborder={0 0 0}]{hyperref}
\usepackage{verbatim}

\usepackage{color}
\definecolor{yellow}{rgb}{1,1,0}
\definecolor{black}{rgb}{0,0,0}
\definecolor{ltcyan}{rgb}{.75,1,1}
\definecolor{red}{rgb}{1,0,0}

% Cite commands I use to abstract away the different ways to reference an
% entry in the bibliography (superscripts, numbers, dates, or author
% abbreviations).  \scite is a short cite that is used immediately after
% when the authors are mentioned.  \lcite is a full citation that is used
% anywhere.  Both should be used right next to the text being cited without
% any spacing.
\newcommand*{\lcite}[1]{~\cite{#1}}
\newcommand*{\scite}[1]{~\cite{#1}}

\newcommand{\etal}{et al.}

\newcommand*{\keyterm}[1]{\emph{#1}}

\newcommand{\fix}[1]{{\color{red}\textsc{[#1]}}}

% Avoid putting figures on their own page.
\renewcommand{\textfraction}{0.05}
\renewcommand{\topfraction}{0.95}
\renewcommand{\bottomfraction}{0.95}

% Make sure this is big enough so that only big figures end up on their own
% page but small enough so that if a figure does have to be on its own
% page, it won't push everything to the bottom because it's not big enough
% to have its own page.
\renewcommand{\floatpagefraction}{.75}

\newenvironment{packed_itemize}{
\begin{itemize}
  \setlength{\topsep}{0pt}
  \setlength{\itemsep}{0pt}
  \setlength{\parskip}{0pt}
  \setlength{\parsep}{0pt}
  \setlength{\partopsep}{0pt}
}{\end{itemize}}

\title{Flexible Analysis Software for Emerging Architectures}

\hyphenation{Para-View Map-Re-duce}

\begin{document}

\sloppy

\maketitle

\begin{abstract}
  We are on the threshold of a transformative change in the basic
  architecture of high-performance computing.  The use of accelerator
  processors, characterized by large core counts, shared but asymmetrical
  memory, and heavy thread loading, is quickly becoming the norm in high
  performance computing.  These accelerators represent significant
  challenges in updating our existing base of software.  An intrinsic
  problem with this transition is a fundamental programming shift from
  message passing processes to much more fine thread scheduling with memory
  sharing.  Another problem is the lack of stability in accelerator
  implementation; processor and compiler technology is currently changing
  rapidly.  In this paper we describe our approach to address these two
  immediate problems with respect to scientific analysis and visualization
  algorithms.  Our approach to accelerator programming forms the basis of
  the Dax toolkit, a framework to build data analysis and visualization
  algorithms applicable to exascale computing.
\end{abstract}

\section{Introduction}
\label{sec:Introduction}

\noindent
Whereas supercomputers throughout the terascale era were almost
unilaterally built from general purpose CPU processors on distributed
memory nodes with a message passing interface, with petascale computing we
are seeing the emerging use of accelerators to meet the execution and
computation requirements of modern leadership-class facilities.  This trend
was kicked off when the Roadrunner supercomputer, first to achieve a
petaFLOP, was built with Cell BE processors\fix{cite}.  At the time,
Roadrunner was an anomaly, but since then many high-performance computers
followed this example.  Today, \fix{X} out of 10 of the fastest supercomputers
are also built with accelerator technology\fix{cite Top500?}.

These accelerators represent a significant departure from how we most often
perform parallel processing.  Computing on the previous generation of high
performance computers involved partitioning data among distributed memory
nodes and running independent processes that pass messages.  However,
accelerators do not work well with such an approach.  Threads on an
accelerator may be grouped in SIMD ``warps,'' can have indeterminate
scheduling, and may be incapable of direct message passing\fix{citeCUDA?}.
Even on processors with more complete and independent cores, taking
advantage of shared memory threads can have its
advantages\lcite{Camp2010,Howison2011}.  Ultimately, our algorithms must
exhibit a more ``pervasive parallelism'' comprising a marked increase in
concurrency and careful data
management\lcite{VisAnalysisExtremeScale,ExascaleRoadMap}. \fix{Check the
  VisAnalysisExtremeScale reference (and similar
  VisualizationandKnowledgeDiscovery in main collection).  I seem to be
  mixing this up with Sean Ahern's report.}

Another problem facing current research and development is the shifting
landscape of the development environment.  The Cell BE processors (and
associated compiler environment) comprising Roadrunner is already
discontinued.  Instead, NVIDIA is aggressively pursing leadership in
accelerator technology for scientific computation with Intel hot on its
heels.  Several compiler technologies such as OpenMP, CUDA, Threaded
Building Blocks, and OpenACC \fix{cite} also compete for multi-threaded
programming.

Our team is creating the Dax toolkit\lcite{Moreland2011:LDAV}, which seeks
to provide a development framework for scientific data analysis and
visualization algorithms for the next generation of high-performance
computers and beyond.  In this paper we document the following features.
\begin{itemize}
\item A general approach to data analysis and visualization algorithm
  development that provides a pervasive parallelism without the complexity
  of parallel programming.
\item An adapter mechanism that encapsulates the change in behavior
  required port the toolkit among devices and compilers.
\end{itemize}

\section{Previous Work}
\label{sec:PreviousWork}

\noindent
To implement algorithms that are configurable with respect to operations,
data structures, and processor idiosyncrasies, Dax relies on
well-established techniques of generic programming\fix{cite}.  Generic
programming uses C++ templates to direct the compiler to specialize a
particular piece of code to alternate implementations.

To maximize the amount of code that has no parallel dependencies, Dax
employs a functor-based execution mode\lcite{Baker2010}.  The intention of
this approach is to write a sequential section of code that operates on a
small section of data as a functional object, and then schedule this
function in parallel independently on large vector components.  The
technique can be thought of as a generalization of the map and reduce
operations in a MapReduce\lcite{MapReduce} framework.

A toolkit with similar goals of simplifying many-core parallel programming
and cross-device porting is Thrust\lcite{Thrust}.  Thrust is a more general
template library that provides a number of generic parallel algorithms.
Thrust provides many of the desired attributes of Dax and is in fact used
to implement many of them.  What differentiates Dax is the simplification
and specialization of its interface.  We can provide generic algorithms and
classes designed specifically for data analysis and visualization as well
as better specialize the data management.

It should be noted that this paper does not cover message passing,
distributed memory, or ``hybrid'' parallelism.  Although this is clearly
important in high-performance computing, the scope of this paper is only on
the shared-memory, many-core parallelism part of this problem.  The
techniques discussed here can be coupled with existing distributed memory
approaches \fix{Ahrens2000, DIY} to complete the hybrid parallelism
required to run concurrently across an entire machine.

\section{Algorithmic Approach}
\label{sec:AlgorithmicApproach}

\noindent
Our basic approach to building algorithms is to build kernels of execution
as functors.  These functors are designed to operate on a small element of
data in a serial and stateless manner.  Because this kernel does work on a
small amount of data, we call it a \keyterm{worklet}.  Around this concept,
we build a system to concurrently schedule these worklets across multiple
elements of a vector.

This approach mirrors that of Baker \etal\scite{Baker2010}.  Both
approaches use C++ templating to generically apply functors in parallel to
vectors of data.  Where our work significantly differs from that of Baker's
is in that we are more focused on the computational geometry problems
related to scientific visualization and data analysis.

Where Baker provides a simple mapping mechanism onto a vector, our system
is designed to provide a variety of parallel scheduling operations.  These
result in worklet types that get scheduled in different ways.  Each worklet
type has a different set of capabilities.  The current set of worklet types
are

\begin{description}[\IEEEsetlabelwidth{\quad}]
\item[Field Map]~\\ The Field Map is functionally equivalent to Baker's
  functional approach.  It applies a worklet operation independently and in
  parallel to each entry in one or more field arrays.
\item[Cell Map]~\\ The Cell Map is similar to the Field Map in functionally
  except that it takes the topology of the mesh into consideration.  The
  worklet is applied to each cell in the mesh and has access to any data,
  including point fields, on that cell.  This map enables operations that
  must interpolate across the cell.
\item[Topology Generator]~\\ The Topology Generator works similarly to the
  Cell Map with the exception that instead of creating a new field, it
  creates a new topology (that is, new cells).  One of the prerequisites of
  invoking a Topology Generator is a classification of how many new cells
  will be generated for each of the input cells.  This could be a constant
  value (which would be typical for a tetrahedralization), or it might be
  different for each input cell (which would be typical for operations like
  threshold or contour) and captured in an array.
\item[Point Reduction]~\\ A Point Reduction operation collects values
  associated with a vertex in the mesh and performs an operation that
  reduces to a single value.  Point Reduction has two primary uses.  First,
  when topology generation creates new vertices, field and other
  information can be reduced to the new point.  Second, a point reduction
  can gather information about all of its incident cells, which can be used
  to interpolate fields created with a Cell Map.
\end{description}

A common theme among all of the worklet types is their behavior of applying
the same operation across many small elements.  This approach is well
proven to be an efficient mechanism to drive many compute core
simultaneously.

\section{Schedule Metaprograms}
\label{sec:ScheduleMetaprograms}

\noindent
\fix{This is where a description of the automatic glue code goes (if Robert
  and Brad decide to put it in this paper).}

\section{Device Adapter}
\label{sec:DeviceAdapter}

\noindent
As multiple vendors vie to provide accelerator-type processors, a great
variance in the computer architecture exists, and we expect to encounter
further changes in the near future.  Likewise, there exist multiple
compiler environments and libraries for these devices.  The most popular of
these include OpenMP, CUDA, and OpenCL (although the latter does not yet
support C++ classes and templates).  These compiler technologies also vary
from system to system.

Consequently, we require our Dax toolkit to easily port from one system to
the next.  At a minimum, we require a base language support, and the
language we choose to support is C++.  The majority of the code in Dax is
constrained to the standard C++ language constructs to minimize the
specialization from one system to the next.

\begin{figure}[htb]
  \centering
  \fix{Diagram here}
  \caption{Diagram of the Dax framework.}
  \label{fig:DaxDiagram}
\end{figure}

Figure~\ref{fig:DaxDiagram} provides an overall diagram of the Dax
framework.  Dax is split into two environments, each with its own API.  The
\keyterm{control environment} is used to describe data, interface with
other libraries, and invoke parallel operations.  The control environment
is designed to run in a single thread within a process.  Parallel
algorithms are run in the \keyterm{execution environment}.  Worklets are
built using the execution environment API, which constrains their
operations to a safe region of data.

The control and execution environments are logically equivalent to the host
and device environments, respectively, in CUDA.  When compiling with CUDA,
these environments mirror each other, but the same logical approach is
taken when no such physical separation exists.

In between these two environments sits the \keyterm{device adapter}.  The
device adapter encapsulates all the specialized code required for running
on a particular device with a particular compiler technology.  The
functionality of the device adapter comprises to main parts: a collection
of parallel algorithms and a module to transfer data between the control
and execution environments.

Each device adapter is expected to implement a set algorithms containing
parallel for, scan, lower bounds (parallel find), stream compact, unique
(remove duplicates).  This list of operations is similar to those suggested
by Blelloch\fix{cite} and also a subset of those provided by the Thrust
library\lcite{Thrust}.  Thrust itself provides a convenient implementation
for device adapter because it itself is portable among devices.  However,
the interface to the device adapter algorithms is independent of Thrust,
and we have an example of a device adapter that can be built without
Thrust.

A device adapter also provides a module to handle the transfer of data
between the control and execution environments.  Unlike other systems like
CUDA and Thrust, which explicitly define separate arrays and copy between
them, the Dax device adapter to allocate and copy data in one monolithic
operation.  The advantage of this approach is that a device adapter for a
system that shared memory between the two environments (such as with
OpenMP) can perform shallow copies to share the data.

With these basic device adapter facilities, we can build a support library
for visualization algorithms.  Because the interface for the device adapter
is independent of the implementation for each device, this support library
can be built in such a way to be portable across many devices.

\section{Generic Array Handle}
\label{sec:GenericArrayHandle}

\noindent
The data model for Dax is deliberately simple.  The basic data container in
Dax is an \keyterm{array handle}.  The array handle acts like a smart
pointer to the data to manage its resource usage.  Array handle objects
maintain a reference count of how many instances point to the same array.

Array handle objects can also allocate and deallocate data as necessary.
For example, when an array handle is used to store the output of an
algorithm, Dax will automatically allocate data in the array to store the
appropriate amount of data.

An array handle object manages data in both the control and execution
environment.  When an array handle is used as input to an algorithm, the
array handle automatically copies data to the execution environment.  This
is done using the data transfer module of the device adapter discussed in
Section~\ref{sec:DeviceAdapter}.  As described previously, if the control
and excution environments can share data, then this data is not physically
copied but rather shared.  The array handle also maintains where data
resides to avoid unnecessary copies.  That is, if data is needed in the
executio environment and is already available in the execution environment,
no copy will be made.  To help applications manage limited memory, the
array handle allows applications to free memory either in the execution
environment or in both environments.

Implicit arrays, containers.

\section{Results}

\noindent
\fix{To round this paper out, we really should have some evidence that our
  proposed method is worthwhile.  I think a straightforward thing to do is
  to run an instance of one of our algorithms (probably Threshold).  We
  could do our standard comparison among processes and the VTK
  implementation.  We could also compare against PISTON as an
  implementation without our introduced overhead.}

\fix{The problem, however, is that we still have not fixed the problem of
  copying implicit arrays in CUDA.  Do we have time to resolve that and run
  the tests?}

\section{Conclusion}
\label{sec:Conclusion}

\noindent

\section*{Acknowledgments}

\noindent
This work was supported in part by the DOE Office of Science, Advanced
Scientific Computing Research, under award number 10-014707, program
manager Lucy Nowell.

Sandia National Laboratories is a multi-program laboratory operated by
Sandia Corporation, a wholly owned subsidiary of Lockheed Martin
Corporation, for the U.S. Department of Energy's National Nuclear Security
Administration.

\bibliographystyle{IEEEtranS}
\bibliography{DaxPDAC2012}

\end{document}
